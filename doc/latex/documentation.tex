\documentclass[paper=a4, fontsize=11pt]{article} % A4 paper and 11pt font size
%\usepackage[utf8x]{inputenc}
%\usepackage[czech]{babel} % English language/hyphenation
\usepackage{amsmath,amsfonts,amsthm} % Math packages
%\usepackage{lipsum} % Used for inserting dummy 'Lorem ipsum' text into the template
%\usepackage{sectsty} % Allows customizing section commands
%\allsectionsfont{\centering \normalfont\scshape} % Make all sections centered, the default font and small caps
\usepackage{fancyhdr} % Custom headers and footers
\pagestyle{fancyplain} % Makes all pages in the document conform to the custom headers and footers
\fancyhead{} % No page header - if you want one, create it in the same way as the footers below
\fancyfoot[L]{} % Empty left footer
\fancyfoot[C]{\thepage} % Empty center footer
\fancyfoot[R]{} % Page numbering for right footer
\renewcommand{\headrulewidth}{0pt} % Remove header underlines
\renewcommand{\footrulewidth}{0pt} % Remove footer underlines
\usepackage{graphicx}
\usepackage{indentfirst}
\usepackage[table]{xcolor}
\usepackage{booktabs}
\newcommand{\ra}[1]{\renewcommand{\arraystretch}{#1}}
\setlength\parindent{20pt} % Removes all indentation from paragraphs - comment this line for an assignment with lots of text
\usepackage{amssymb}
% Margins
\topmargin=-0.45in
\evensidemargin=0.0in
\oddsidemargin=-0.5in
\hoffset=1.0in
\textwidth=5.5in
\textheight=9.0in
\headsep=0.25in
%----------------------------------------------------------------------------------------
%	TITLE SECTION
%----------------------------------------------------------------------------------------

\newcommand{\horrule}[1]{\rule{\linewidth}{#1}} % Create horizontal rule command with 1 argument of height

\title{	
\normalfont \normalsize 
\textsc{Norwegian University of Science and Technology} \\ [25pt] % Your university, school and/or department name(s)
\horrule{0.5pt} \\[0.4cm] % Thin top horizontal rule
\huge Using Minimax with Alpha-Beta pruning to play Quarto\\ % The assignment title
\horrule{2pt} \\[0.5cm] % Thick bottom horizontal rule
}

\author{Jan Bednarik\\Tomas Dohnalek} % Your name

\date{\normalsize\today} % Today's date or a custom date

\begin{document}

\maketitle % Print the title

\section{Introduction}
\section{State evaluation function}
\section{Results}

In this section we present results of some matches between our-implemented players and also results of our special player Minimax-0 in tournament.

\subsection{Random versus Novice}
You can see the results if 100 games in figure \ref{fig:random-novice}. 
In half of the games started Random, in the other half it was Novice.



\begin{figure}[ht]
    \begin{minipage}[c]{0.40\linewidth}
        \centering
        \ra{1.3}
        \begin{tabular}{cll}
            \toprule
            \textcolor{red!100}{$\bullet$} & Random wins & 3       \\
            \textcolor{blue!100!yellow!100!red!80}{$\bullet$} & Novice wins & 97      \\  
            \bottomrule
        \end{tabular}
    \end{minipage}
    \begin{minipage}[c]{0.60\linewidth}
        \centering
        \includegraphics[scale=0.35]{img/random-novice.pdf}
    \end{minipage}
    \caption{100 games of Random against Novice.}
    \label{fig:random-novice}
\end{figure}



\subsection{Novice versus Minimax-3}
You can see the results if 20 games in figure \ref{fig:novice-minimax3}. 
In half of the games started Novice, in the other half it was Minimax-3.

\begin{figure}[ht]
    \begin{minipage}[c]{0.40\linewidth}
        \centering
        \ra{1.3}
        \begin{tabular}{cll}
            \toprule
            \textcolor{red!100}{$\bullet$} & Novice wins & 1       \\
            \textcolor{blue!100!yellow!100!red!80}{$\bullet$} & Minimax-3 wins & 18      \\  
            \textcolor{gray!100}{$\bullet$} & Draws & 1      \\  
            \bottomrule
        \end{tabular}
    \end{minipage}
    \begin{minipage}[c]{0.60\linewidth}
        \centering
        \includegraphics[scale=0.35]{img/novice-minimax3.pdf}
    \end{minipage}
    \caption{20 games of Novice against  Minimax-3.}
    \label{fig:novice-minimax3}
\end{figure}


\subsection{Minimax-3 versus Minimax-4}
You can see the results if 20 games in figure \ref{fig:minimax3-minimax4}. 
In half of the games started Minimax-3, in the other half it was Minimax-4.


\begin{figure}[ht]
    \begin{minipage}[c]{0.40\linewidth}
        \centering
        \ra{1.3}
        \begin{tabular}{cll}
            \toprule
            \textcolor{red!100}{$\bullet$} & Minimax-3 wins & 2       \\
            \textcolor{blue!100!yellow!100!red!80}{$\bullet$} & Minimax-4 wins & 9      \\  
            \textcolor{gray!100}{$\bullet$} & Draws & 9      \\  
            \bottomrule
        \end{tabular}
    \end{minipage}
    \begin{minipage}[c]{0.60\linewidth}
        \centering
        \includegraphics[scale=0.35]{img/minimax3-minimax4.pdf}
    \end{minipage}
    \caption{20 games of Minimax-3 against Minimax-4.}
    \label{fig:minimax3-minimax4}
\end{figure}

\subsection{Minimax in tournament}
We have played 20 games with our Minimax-4 player against Dominik's and Pablo's player.
Results can be seen in figure \ref{fig:tournament}.
In half of the games started our player, in the other half it was player of Dominik and Pablo.

\begin{figure}[ht]
    \begin{minipage}[c]{0.40\linewidth}
        \centering
        \ra{1.3}
        \begin{tabular}{cll}
            \toprule
            \textcolor{red!100}{$\bullet$} & Dominik's and Pablo's wins & 0       \\
            \textcolor{blue!100!yellow!100!red!80}{$\bullet$} & Our Minimax-4  wins & 18      \\  
            \textcolor{gray!100}{$\bullet$} & Draws & 2      \\  
            \bottomrule
        \end{tabular}
    \end{minipage}
    \begin{minipage}[c]{0.60\linewidth}
        \centering
        \includegraphics[scale=0.35]{img/tournament.pdf}
    \end{minipage}
    \caption{20 games of Dominik's and Pablo's Minimax-4 against our Minimax-4.}
    \label{fig:tournament}
\end{figure}

Unfortunately we could not play against Marc's and Valerio's player because they were not able to implement network module for this project in time.

\section{Reflection}
In this section we will reflect our experience with programming for the tournament and also the participation in tournament itself.

\subsection{Experience of programming for tournament}
During designing architecture of our program, we had a short  meeting with other teams and we have decided to use client-server architecture. 
We did not design any protocol at that moment.

As Dominik and Pablo have finished their implementation early, they suggested that they can create simple server as an extension to their application. 
They have made their own protocol and implemented the server and example of client in Java. 
As we use C++/Qt we did not have to implement client from scratch, we could use some Qt classes that let us work over network very easily. 

The biggest issue was connected with fact that our architecture was not designed to play repeated games (as we intended to run repeated games in simple loop from shell) and we had to redesign and reimplement some parts of our code. We have also encountered issues related to networking and understanding the protocol.

\subsection{Participation in tournament}
We have created special player called \emph{Minimax-0} which had no fixed depth of alphabeta search and use variable depth instead. As game continues we increase the depth because the search space is getting rapidly smaller after each turn. From 8th move we were able to use Minimax-8 and so we could explore the search space entirely. In the beginning of tournament we have agreed with our opponents that we would not player and use Minimax-4 instead.

As Marc and Valerio were not able to finish they program, we played only against Dominik's and Pablo's player. Our client has won most of the games and we were of couse satisfied with this results. After the tournament we have discussed our strategy with our opponents and we figured out that the main difference is of course in evaluation function. Ours was probably much more powerful.

\section{Summary}
Goal of this project was to create a working program playing game called Quarto. We have implemented several different kinds of players such as Random, Novice, Minimax with different depths of search, Human player and also Remote player who can be some other application. We have participated in tournament with two other teams; unfortunately one team was not able to deliver their application so we had to play only against once client. 

Implemented Minimax player was able to easily beat all other opponents including some applications we have explored on the Internet. It would be nice to compare our Minimax player with even more applications so we can learn if used state-evaluation method was powerful enough.

%-----------------------------------------------
\begin{flushleft}
%\bibliography{literatura} % viz. literatura.bib..
%\bibliographystyle{czplain}
\end{flushleft}
\end{document}
